\documentclass{article}
\usepackage{amssymb,amsmath, verbatim}
%\usepackage{titlesec}  
\usepackage{epsfig}
%\usepackage{placeins}
\usepackage[margin=2cm]{geometry}
%\usepackage{fancyhdr}
%\pagestyle{fancy}
%\renewcommand{\headrulewidth}{0pt}
%\renewcommand{\footrulewidth}{1pt}

\usepackage{graphicx}
\usepackage[T1]{fontenc}
\usepackage{listings}
\usepackage{bm}

%\usepackage{names}

\DeclareMathOperator*{\argmax}{\arg\!\max}
\DeclareMathOperator*{\argmin}{\arg\!\min}

\newcommand{\pdiff}[2]{
        \frac{\partial #1}{\partial #2}
}

\newcommand{\diff}[2]{
        \frac{d #1}{d #2}
}


% Footer
%\lfoot{\thepage}
%\rfoot{\vspace{-.8cm} Michael Stewart, Andrea Corredor, Elahe Rahimtoroghi
%\\\vspace{.14cm}
%\footnotesize mijstewa@ucsc.edu, acorredo@ucsc.edu, elahe@soe.ucsc.edu}


                 
%\titleformat{\subsection}[runin]{}{}{}{}[] % makes subsections not appear on their own lines


%Header
\newcommand{\header}[5]{
        \begin{minipage}[h!]{0.63\textwidth}
                \centering
                { \LARGE \textbf{ \textsc{#1} } }
        \end{minipage}
        \begin{minipage}[h!]{0.37\textwidth}
                \centering
                {#2}\\
                {#3}\\
        \end{minipage}
}

\begin{document}

\header{Machine Learning HW \#3}
       {Michael Stewart\\Andrea Corredor\\Elahe Rahimtoroghi}
       {11/5/2013}
\\\\


\section{Decision Trees}

\subsection*{a: Pure children}
The right child is pure because all its data points belong to the same class (all have label 1).

\subsection*{b: Impurity if $x_2$ at the left node}
\begin{align*}
\text{Average impurity} = \frac{\left( n_1 \text{impurity}(p_1) + n_2 \text{impurity}(p_2) \right)}{n} \\
\text{impurity}(p) = 2p(1-p) \\  
\text{Let } n_1 \text{correspond to } x_2 = 0 \text{ and } n_2 \text{ to } x_2 = 1 \\
n1 = 2 \\
p1 = 1/2 \\
n2 = 2 \\
p2 = 1/2 \\
n = 4 \\
\text{Average impurity} = \frac{\left( 2*2*\frac{1}{2}\frac{1}{2}  + 2*2* \frac{1}{2}\frac{1}{2} \right)}{4} \\
\text{Average impurity} = \frac{1}{2}
\end{align*}
\subsection*{c: Impurity if $x_3$ at root}
\begin{align*}
\text{Let } n_1 \text{correspond to } x_3 = 0 \text{ and } n_2 \text{ to } x_3 = 1 \\
n1 = 1 \\
p1 = 1 \\
n2 = 3 \\
p2 = 1/3 \\
n = 4 \\
\text{Average impurity} = \frac{\left( 1*2*1*0  + 3*2* \frac{1}{3}\frac{2}{3} \right)}{4} \\
\text{Average impurity} = \frac{1}{3}
\end{align*}

\section{Artifical Neural Network}
Assume that the nodes do not have bias terms, and the initial weights are all $0$'s. \\
$\eta = 0.1$, training example $x_1 = 1, x_2 =2, t=1$. Show the $a_i,z_i$ and $\delta_i$ values for each non-input node, and the new weights after the backprop update.
\subsection*{Backpropagation with logistic sigmoid}
Going forward to compute all $a_j$ and $z_j$: \\
$x_1 = a_1 = z_1 = 1,~~x_2 = a_2 = z_2 =  2$ \\
$a_3 = \sum_{i=1}^{2} w_{3i} z_{i} =  0*1 + 0*2 = 0   $ \\
$z_3 = \sigma(a_3) = \frac{1}{1 + e^0} = \frac{1}{2}$ \\
$a_4 = \sum_{i=1}^{2} w_{4i} z_{i} =  0*1 + 0*2 = 0   $ \\
$z_4 = \sigma(a_4) = \frac{1}{2}$ \\
$a_5 = \sum_{i=3}^{4} w_{5i} z_{i} =  0*\frac{1}{2} + 0*\frac{1}{2} = 0   $ \\
$z_5 = a_5 = 0$ \\
\\
Computing $\frac{\partial E}{\partial a_k}$ at the output node: \\
$\frac{\partial E}{\partial a_{5}} = \left( a_5 - t \right) = (0 - 1) = -1$ \\
\\
Using Equation (5) to compute the other $\frac{\partial E}{\partial a_j}$ , working backwards through the network: \\
\begin{align*}
\frac{\partial E}{\partial a_j} = \left( \sum_{k \in U_j} \frac{\partial E}{\partial a_k} w_{kj}  \right) z_j (1-z_j)
\end{align*}
\begin{align*}
\frac{\partial E}{\partial a_4} &= \left( \frac{\partial E}{\partial a_5}*w_{5,4} \right)z_4(1-z_4) \\
&= (-1*0)*0.5*0.5 \\
&= 0
\end{align*}
\begin{align*}
\frac{\partial E}{\partial a_3} &= \left( \frac{\partial E}{\partial a_5}*w_{5,3} \right)z_3(1-z_3) \\
&= (-1*0)*0.5*0.5 \\
&= 0
\end{align*}
Thus, we find $\delta_4 = 0,~\delta_3 = 0$
\\
Using Equation (6) to compute the derivative of the error with respect to each weight in the network: \\
Eq. (6) $\frac{\partial E}{\partial w_{ji}} = \delta_j z_i$
\begin{align*}
\frac{\partial E}{\partial w_{5,3}} &= \delta_5 z_3 = -1*\frac{1}{2} = -\frac{1}{2} \\
\frac{\partial E}{\partial w_{5,4}} &= \delta_5 z_4 = -1*\frac{1}{2} = -\frac{1}{2} \\
\frac{\partial E}{\partial w_{3,1}} &= \delta_3 z_1 = 0*1 = 0 \\
\frac{\partial E}{\partial w_{3,2}} &= \delta_3 z_2 = 0*2 = 0 \\
\frac{\partial E}{\partial w_{4,1}} &= \delta_4 z_1 = 0*1 = 0 \\
\frac{\partial E}{\partial w_{4,2}} &= \delta_4 z_2 = 0*2 = 0
\end{align*}
\\
\\
Finally, update each $w_{ji}$ weight to $w{ji} - \eta \frac{\partial E}{\partial w_{ji}}$ where $\eta$ is the learning rate:
\begin{align*}
 w_{5,3} &= w_{5,3} - 0.1*-\frac{1}{2} = 0.05\\
 w_{5,4} &= w_{5,4} - 0.1*-\frac{1}{2} = 0.05\\
 w_{3,1} &= w_{5,3} - 0.1*0 = 0\\
 w_{3,2} &= w_{5,3} - 0.1*0 = 0\\
 w_{4,1} &= w_{5,3} - 0.1*0 = 0\\
 w_{4,2} &= w_{5,3} - 0.1*0 = 0\\
\end{align*}


\subsection*{Backpropagation with hyperbolic tangent sigmoid}
Now $\sigma(a) = tanh(a)$, which changes equation (5) as the derivative of $\frac{z_j}{a_j}$ is now $1 - \sigma(a)^2$. Since Eq (6) depends on Eq (5), the former also changes. 
The updated equations are:
\begin{align*}
\frac{\partial E}{\partial a_j} &= \delta_j = \left( \sum_{k \in U_j} \frac{\partial E}{\partial a_k} w_{kj}  \right) \left( 1 - z_j^2 \right) \\
\frac{\partial E}{\partial w_{ji}} &= \delta_j z_i  Eq. (6)
\end{align*}

Going forward to compute all $a_j$ and $z_j$: \\
$x_1 = a_1 = z_1 = 1,~~x_2 = a_2 = z_2 =  2$ \\
$a_3 = \sum_{i=1}^{2} w_{3i} z_{i} =  0*1 + 0*2 = 0   $ \\
$z_3 = \sigma(a_3) = \frac{e^0 - e^0}{e^0 + e^0} = 0$ \\
$a_4 = \sum_{i=1}^{2} w_{4i} z_{i} =  0*1 + 0*2 = 0   $ \\
$z_4 = \sigma(a_4) = 0 $ \\
$a_5 = \sum_{i=3}^{4} w_{5i} z_{i} =  0*0 + 0*0 = 0   $ \\
$z_5 = a_5 = 0$ \\
\\
Computing $\frac{\partial E}{\partial a_k}$ at the output node: \\
$\frac{\partial E}{\partial a_{5}} = \left( a_5 - t \right) = (0 - 1) = -1$ \\
\\
Using Equation (5) to compute the other $\frac{\partial E}{\partial a_j}$ , working backwards through the network: \\
\begin{align*}
\frac{\partial E}{\partial a_j} = \left( \sum_{k \in U_j} \frac{\partial E}{\partial a_k} w_{kj}  \right)(1-z_j^2)
\end{align*}
\begin{align*}
\frac{\partial E}{\partial a_4} &= \left( \frac{\partial E}{\partial a_5}*w_{5,4} \right)(1-z_4^2) \\
&= (-1*0)*(1-0) \\
&= 0
\end{align*}
\begin{align*}
\frac{\partial E}{\partial a_3} &= \left( \frac{\partial E}{\partial a_5}*w_{5,3} \right)(1-z_3^2) \\
&= (-1*0)*(1-0)\\
&= 0
\end{align*}
Thus, we find $\delta_4 = 0,~\delta_3 = 0$
\\
Using Equation (6) to compute the derivative of the error with respect to each weight in the network: \\
Eq. (6) $\frac{\partial E}{\partial w_{ji}} = \delta_j z_i$
\begin{align*}
\frac{\partial E}{\partial w_{5,3}} &= \delta_5 z_3 = -1*0 = 0 \\
\frac{\partial E}{\partial w_{5,4}} &= \delta_5 z_4 = -1*0 = 0 \\
\frac{\partial E}{\partial w_{3,1}} &= \delta_3 z_1 = 0*1 = 0 \\
\frac{\partial E}{\partial w_{3,2}} &= \delta_3 z_2 = 0*2 = 0 \\
\frac{\partial E}{\partial w_{4,1}} &= \delta_4 z_1 = 0*1 = 0 \\
\frac{\partial E}{\partial w_{4,2}} &= \delta_4 z_2 = 0*2 = 0
\end{align*}
\\
\\
Finally, update each $w_{ji}$ weight to $w{ji} - \eta \frac{\partial E}{\partial w_{ji}}$ where $\eta$ is the learning rate:
\begin{align*}
 w_{5,3} &= w_{5,3} - 0.1*0 = 0\\
 w_{5,4} &= w_{5,4} - 0.1*0 = 0\\
 w_{3,1} &= w_{5,3} - 0.1*0 = 0\\
 w_{3,2} &= w_{5,3} - 0.1*0 = 0\\
 w_{4,1} &= w_{5,3} - 0.1*0 = 0\\
 w_{4,2} &= w_{5,3} - 0.1*0 = 0\\
\end{align*}

\section{XOR and Majority vote datasets}

\subsection*{a: Tree complexity}
The tree obtained using the XOR dataset had 15 nodes, and the root's depth is 3.
The tree for the majority vote dataset had 11 nodes, and the root's depth is 3.
Both trees are reasonably interpretable. The XOR tree simply tests all 8 possible combinations of the first 3 features.
The majority decision tree looks at cases where $x_2$ and $x_1$ are the same, and cases where they are not it looks at $x_3$ to be the tie breaker. 

%If xor tree captures all 3 cases, why is it getting some of the predictions wrong???

When we adjust the data file for the xor problem so that each combination of the values of the three relevant features (ie. 000, 001, 010, etc.) appears equally often, the tree we obtain is considerably larger with 101 nodes, 51 leaves and root depth 8. The tree does not single out the first 3 features as being the ones that generate the label. 

%WHY??

\subsection*{b: Neural Networks}

XOR takes approximately 50-60 epochs to converge with 3 hidden layers. 
Majority vote takes 4 epochs to converge with 2 hidden layers.

\subsection*{c: SVMs}

All 3 cases 0\% classification error for the majority vote dataset. 

XOR dataset with exponent 1: 75\% - 78\% correctly classified.
XOR with exponent 2 without lower order terms: 50\% -65\% correctly classified. 
XOR with exponent 2 with lower order terms: 50\% - 65\% correctly classified. 

\end{document}
